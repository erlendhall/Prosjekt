\documentclass[Main/main.tex]{subfiles}

\begin{document}
\newpage
\section{Results}

%Data and results as they are

%Data: Observations, instrument output, measurements, curves, diagrams, raw data
%Results: Mathematical manipulated data


\begin{itemize}
    \item Findings? Effect of position of needle, concentration, placement of substrate, covering the substrate
    \item Identification - XRD, SEM, Nanoscratch
    \begin{itemize}
        \item SEM - particle size, homogenity, surface roughness
        \item XRD - phases present
    \end{itemize}
    \item Electrochemical - Impedance, CV and GC
    \item Other results?
    \item \ce{MnO2}, \ce{LiMn2O4}, \ce{LiMn_{1.5}Ni_{0.5}O4}
\end{itemize}

\subsection{Liquid injection system}

\textbf{Table of different parameters and their effect?}
Position of needle - no visible effect. Might be a chance of clogging if the needle is very far down.

Concentration of precursors -> Particle size and weight of deposited material

\subsection{Characterizing - SEM and XRD}

\subsubsection{\ce{MnO2}}
	
Put SEM and XRD together?





\subsubsection{\ce{LiMn2O4}}







\subsubsection{\ce{LiMn_{1.5}Ni_{0.5}O4} - LNMO}




\subsection{Electrochemical analysis}

\subsubsection{\ce{MnO2}}



\subsubsection{\ce{LiMn2O4}}


\subsubsection{\ce{LiMn_{1.5}Ni_{0.5}O4} - LNMO}


\subsection{\textbf{OR}}

\subsubsection{Impedance}

\subsubsection{Cyclic Voltametry}

\subsubsection{Galvnostatic Cycling}















\end{document}