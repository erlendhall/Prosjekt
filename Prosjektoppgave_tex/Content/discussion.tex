\documentclass[Main/main.tex]{subfiles}

\begin{document}

\chapter{Conclusion}

The liquid injection system presented here is far from perfect for the synthesis of cathode materials due to a lack of precision regarding the synthesised oxides and several of said oxides yielding cathodes with capacitive behaviour. Nevertheless, the system has proved capable of synthesising some of the desired oxides which adhered to a steel substrate and was thus able to be used as the cathode in a cell without further modification. The morhpology seemed to mainly consist of hollow, porous spheres which has been reported as advantageous in cathodes due to factors such as rapid diffusion and enhanced rate capability.

Indeed, the electrochemical analyses showed that chemical reactions were occurring in most of the cells, indicating that the cells can be used in batteries, preferably after some further modifications. The low resistance shown by the EIS is very promising for the notion of synthesizing cathode materials without the need for a binder.

In general, the synthesised cathodes exhibit lower internal resistance than commercial LMO cathodes, but it is difficult to read the electrochemical reactions.

Further improvements and research could be done by achieving greater control of the deposition parameters, ideally obtaining similar deposited masses for consecutive syntheses. Furthermore, it would be interesting to see cathodes with even more active material, possibly achieved by injecting a greater volume during liquid injection. Lastly, another attempt should be made at synthesising \ce{MnO2} by lowering the temperature.

In conclusion, this work has expanded the understanding of the liquid injection system in vacuum and the effect of different parameters, and proved this process to be a candidate for the synthesis of low-resistance cathode materials. 


\end{document}